
\documentclass[a4paper,12pt]{report}
\usepackage{etex}
% \usepackage[latin1]{inputenc}
 \usepackage[T1]{fontenc}
\usepackage{makeidx}
\usepackage{lettrine}
\usepackage{yfonts}
%\usepackage{manyfoot}

\usepackage[svgnames]{pstricks}

%\usepackage[dvipsnames]{pstricks}
 \usepackage{pstricks,pst-plot,pstricks-add,pst-math,pst-node}
 \usepackage{titlesec}

 


\usepackage{url}
\usepackage{hyperref}
\hypersetup{colorlinks=true,urlcolor=blue}
\usepackage{breakurl}
%\usepackage{oldnbs11}
%\usepackage{ospecial}
\usepackage{palatino}
\usepackage{listings}
\usepackage{graphics}
%\usepackage{aeguill}
\usepackage{lmodern}
%\usepackage[english,greek,french]{babel}
\usepackage[english]{babel}

%\languageattribute{greek}{polutoniko}

\usepackage{eurosym}
%\usepackage[hmargin={1.8cm,1.9cm},top=1.6cm,bottom=2.5cm]{geometry}
\usepackage[hmargin={1.3cm,1.7cm},top=0.8cm,bottom=3cm]{geometry}
\setlength{\headheight}{37pt}
%\setlength{\headsep}{4cm}%\pagestyle{empty}

\usepackage{textcomp}
\usepackage{float, color, soul, ulem} 
\usepackage[dvips]{graphicx}
\usepackage {xcolor}
\usepackage{enumerate}
\usepackage{amsmath}
\usepackage{amsfonts,verbatim,mathdots}
\usepackage{amssymb}
\usepackage{amsthm}
\usepackage{mathrsfs}
\usepackage{stmaryrd}
\usepackage{epigraph}
\usepackage{ifthen}
\usepackage{dsfont}
\usepackage{animate}

\usepackage{subfig}
\usepackage{amscd}

\usepackage{booktabs} % for much better looking table
\usepackage{array} % for better arrays (eg matrices) in maths

\usepackage{epsfig}
\usepackage{tikz}



\floatplacement{figure}{H}
\definecolor{grey}{rgb}{0.95,0.95,0.95}
\lstloadlanguages{matlab}

\renewcommand{\lstlistingname}{Code}
%\lstset{basicstyle=\small,  breaklines=true, numbersep=7pt, numbers=left, language=matlab, frame=single, backgroundcolor=\color{grey}}
%\def\Motscle{emph={\lst@keywords}}
   % \expandafter\lstset\expandafter{%
      %\Motscle}
    \definecolor{Ggris}{rgb}{0.45,0.48,0.45}
 
    \lstset{emphstyle=\rmfamily\color{blue},
    basicstyle=\small,
    keywordstyle=\ttfamily,
    commentstyle=\color{red},
    numberstyle=\tiny\color{red},
    numbers=left,
    numbersep=10pt,
    lineskip=0.7pt,
    showstringspaces=false,
    %frame=tBlR ,rulesep =1mm ,framesep =5mm,framerule =2pt,
%xrightmargin =5mm,xleftmargin =5mm,
breaklines=true,backgroundcolor=\color{grey}}
 
 
\usepackage{color} %red, green, blue, yellow, cyan, magenta, black, white
\definecolor{mygreen}{RGB}{28,172,0} 
\definecolor{mylilas}{RGB}{170,55,241}


\lstset{language=Matlab,%
    %basicstyle=\color{red},
    breaklines=true,%
    morekeywords={matlab2tikz},
    keywordstyle=\color{blue},%
    morekeywords=[2]{1}, keywordstyle=[2]{\color{black}},
    identifierstyle=\color{black},%
    stringstyle=\color{mylilas},
    commentstyle=\color{mygreen},%
    showstringspaces=false,%without this there will be a symbol in the places where there is a space
    numbers=left,%
    numberstyle={\tiny \color{black}},% size of the numbers
    numbersep=9pt, % this defines how far the numbers are from the text
    emph=[1]{for,end,break},emphstyle=[1]\color{red}, %some words to emphasise
    %emph=[2]{word1,word2}, emphstyle=[2]{style},    
}
 
 



  \newboolean{footmark}
  \newcounter{savedfootnote}
  \newcommand*{\footmark}{%
    \unless \iffootmark
      \global \footmarktrue
      \setcounter{savedfootnote}{\value{footnote}}%
    \fi
    \footnotemark
}

  \newcommand*{\foottext}{%
    \iffootmark
      \global \footmarkfalse
      \setcounter{footnote}{\value{savedfootnote}}%
    \fi
    \refstepcounter{footnote}%
    \footnotetext
}
%INFO
\def\dem{\gi{D�monstration} : \\}
\def\rem{\gi{Remarque} : \\}\def\rems{\gi{Remarques} : }
\def\th{\gi{Th��or��me} :\:\\}\DeclareMathOperator{\loc}{loc}\DeclareMathOperator{\Rec}{Rec}\DeclareMathOperator{\Loc}{Loc}\DeclareMathOperator{\true}{true}\DeclareMathOperator{\false}{false}


%   REDEFINITIONS :
\newfont{\cmmd}{rsfs10 at 10pt}
\newfont{\pcmmd}{rsfs10 at 7pt}
\DeclareMathOperator{\tr}{Tr} \DeclareMathOperator{\sh}{sh}
\DeclareMathOperator{\adh}{adh}
\DeclareMathOperator{\dett}{Det}\DeclareMathOperator{\coshyp}{ch}\DeclareMathOperator{\Det}{Det}
\DeclareMathOperator{\Quad}{Quad}
\DeclareMathOperator{\Conv}{Conv}\DeclareMathOperator{\Extr}{Extr}
\DeclareMathOperator{\Id}{Id}
\DeclareMathOperator{\da}{da}
\DeclareMathOperator{\codim}{codim}
\DeclareMathOperator{\gram}{gram}
\DeclareMathOperator{\grad}{grad}
\DeclareMathOperator{\sinhyp}{sh}
\DeclareMathOperator{\tang}{tg}\DeclareMathOperator{\Ker}{Ker}
\DeclareMathOperator{\Img}{Im}\DeclareMathOperator{\Arctg}{Arc\,tg}
\DeclareMathOperator{\rg}{rg}\DeclareMathOperator{\Imm}{Im}\DeclareMathOperator{\Ree}{Re}
\DeclareMathOperator{\Arccos}{Arc\,cos}\DeclareMathOperator{\Log}{Log}\DeclareMathOperator{\Ln}{Ln}\DeclareMathOperator{\argsh}{argsh}
\DeclareMathOperator{\Arcsin}{Arc\,sin} \DeclareMathOperator{\ch}{ch}
\DeclareMathOperator{\ic}{i}\DeclareMathOperator{\iii}{i}
\DeclareMathOperator{\sg}{sg}\DeclareMathOperator{\Sp}{Sp}
\DeclareMathOperator{\Card}{Card}
\DeclareMathOperator{\gr}{gr}
\newcommand{\moins}[1]{\smallsetminus\left\{#1\right\}}
\newcommand{\acco}[1]{\left\{#1\right\}} % donne {.......}
\newcommand{\pare}[1]{\left(#1\right)}
\newcommand{\croc}[1]{\left[#1\right]}
\newcommand{\abs}[1]{\left\lvert#1\right\rvert}
\newcommand{\trema}[1]{\"#1}
\newcommand{\virg}{\raise1pt\hbox{,}\ }
\newcommand{\pnt}{\raise1pt\hbox{.}\ }
\newcommand{\pe}[1]{\left\lfloor#1\right\rfloor}
\newcommand{\dint}{\displaystyle\int}
\newcommand{\dsum}{\displaystyle\sum}
\newcommand{\dprod}{\displaystyle\prod}
\newcommand{\qqs}[2]{\left(\forall#1\in#2\right)\:}
\newcommand{\ile}[2]{\left(\exists#1\in#2\right)\:}
\newcommand{\ileu}[2]{\left(\exists\:!#1\in#2\right)\:}
\def\ntb#1{|\!|\!|#1|\!|\!|}
\newcommand{\bqqs}[2]{\big(\forall#1\in#2\big)\:}
\newcommand{\bqqsn}[2]{\big(\forall#1\notin#2\big)\:}
\newcommand{\Bqqs}[2]{\Big(\forall#1\in#2\Big)\:}
\newcommand{\bile}[2]{\big(\exists#1\in#2\big)\:}
\newcommand{\bileu}[2]{\big(\exists\:!#1\in#2\big)\:}
\newcommand{\ooo}[1]{\raisebox{-4pt}{\scalebox{#1}{$\text{O}$}}}
\newcommand{\cdoot}[1]{\raisebox{-2pt}{\scalebox{#1}{$\cdot$}}}
\newcommand{\es}{\teh{4}{\scalebox{1.3}{\smiley}}{=}}
\DeclareMathOperator{\pgcd}{pgcd}
\DeclareMathOperator{\Vect}{Vect}
\DeclareMathOperator{\Mat}{Mat}
\DeclareMathOperator{\Ima}{Im}
\DeclareMathOperator{\diag}{diag}
\DeclareMathOperator{\Var}{Var}
\DeclareMathOperator{\spe}{sp}
\DeclareMathOperator{\sig}{sg}
\DeclareMathOperator{\Cov}{Cov}
\def\suit{\rightsquigarrow}
\def\TT{\mathscr{T}}\def\PO{\mathscr{P}(\Omega)}\def\LL{P}

\def\cchi{\raisebox{2pt}{$\chi$}}
\def\idee{\raisebox{-4pt}{\bclampe}}
\def\beuh{\raisebox{-5pt}{\bcsmmh}}
\def\hihi{\raisebox{-5pt}{\bcsmbh}}
\def\attention{\raisebox{-4pt}{\bcdanger}}

 \newcommand{\grec}[1]{\selectlanguage{greek}#1\selectlanguage{francais}}
\def\cchi{\raisebox{2pt}{$\chi$}}\def\suit{\rightsquigarrow}
\def\jfrac#1#2{\raisebox{2pt}{$#1$}/\raisebox{-2pt}{$#2$}}
\def\dbind#1{_{_{#1}}}  %Passage direct en indice d'indice

\def\bcdot{\raisebox{-2pt}{\scalebox{2}{$\cdot$}}}\def\bcdotd{\raisebox{-3pt}{\scalebox{2}{$\cdot$}}}
%\def\mp{\vskip 0,2cm} 
\def\bp{\vskip 0,4cm} 
\def\Esp{{\mathbb E}}
\def\Pr{{\mathbb P}}
\def\EE{{\mathbf E}}
\def\R{{\mathbb R}}
\def\C{{\mathbb C}}
\def\Z{{\mathbb Z}}
\def\Q{{\mathbb Q}}
\def\N{{\mathbb N}}
\def\K{{\mathbb K}}

\def\A{{\mathcal A}}
\def\B{{\mathcal B}}

\def\L{{\mathcal L}}
\def\M{{\mathcal M}}
\def\gen{{\mathcal G}}
\def\diff{{\mathrm d}}

\def \ind{\mbox{1\hspace{-.30em}I}}
\def\l{\lambda}\def\m{\mu}\def\a{\alpha}\def\b{\beta}\def\n{\nu}\def\La{\Lambda}\def\g{\gamma}\def\x{\xi}\def\o{\omega}\def\O{\Omega}\def\vf{\varphi}\def\d{\delta}\def\G{\Gamma}\def\r{\varrho}\def\p{\psi}\def\si{\sigma}\def\D{\Delta}\def\t{\theta}\def\e{\varepsilon}\def\T{\Theta}\def\vt{\vartheta}\def\z{\zeta}\def\F{\Phi}

%D�finition

\def \s{\stackrel{\text{d�ef}}{=}}
\newcommand{\egaldef}{\stackrel{{\scriptscriptstyle{def}}}{=}}
\def\etplus{\raisebox{-3pt}{\text{\small*}}\raisebox{0.5pt}{\text{\tiny+}}}   %Pour R^{+*}

%Function
     \newcommand{\fonc}[4]{\left\{\begin{array}{c}#1\longrightarrow #2\\ #3 \longmapsto #4\end{array}\right}
    \newcommand{\foncc}[4]{\left\{\begin{array}{ccc}#1 & \longrightarrow & #2 \\ #3& \longmapsto & #4\end{array}\right.}
     \newcommand{\foncrap}[4]{#1 \ni #2 \longmapsto #3 \in #4}
  \newcommand{\fonccourte}[2]{#1 \longmapsto #2}
   \newcommand{\foncfleche}[2]{#1 \rightarrow #2}
    \newcommand{\foncsanspoint}[4]{#1 \ni #3 \longmapsto #4 \in #2}
   \newcommand{\fun}[4]{\begin{array}{l r c l}
& #1 & \longrightarrow & #2 \\
    & #3 & \longmapsto & #4 \end{array}}
   
   

\def\dem{\gi{D��monstration}\\}
\def\tp{\dot{\t}}\def\tpp{\ddot{\t}}\def\jfrac#1#2{\raisebox{2pt}{$#1$}/\raisebox{-2pt}{$#2$}}


\def\Mnr{\mathfrak{M}_n(\R)}\def\Mnc{\mathfrak{M}_n(\C)}\def\Mnk{\mathfrak{M}_n(\K)}
\def\Mnur{\mathfrak{M}_{n,1}(\R)}\def\Mnuc{\mathfrak{M}_{n,1}(\C)}\def\Mnuk{\mathfrak{M}_{n,1}(\K)}

 \newcommand{\bsum}{\ensuremath\mathop{\scalebox{1.4}{$\displaystyle\sum$}}}
\newcommand{\bint}{\ensuremath\mathop{\scalebox{1.4}{$\displaystyle\int$}}}
\newcommand{\oo}{\raisebox{-2pt}{\scalebox{1.5}{$\text{O}$}}}
%\fancyfoot{{\tiny{Cheikh Tour�}} \hskip 0.2cm\hrulefill\hspace{0.4cm}page 1}
%\lhead{\textbf{\begin{huge}\end{huge}}} 
 %\chead{\textbf{\begin{LARGE} ENS Cachan \end{LARGE}}}
%  \rhead{\textbf{2015}}
%\renewcommand{\headrulewidth}{0pt}
\renewcommand{\labelenumi}{\textbf{\Roman{enumi}-}}

%\renewcommand{\Frlabelitemi}{$\bullet$} for french language

%\renewcommand{\Frlabelitemii}{$\star$}
%\renewcommand{\Frlabelitemiii}{$\-$}
\renewcommand{\labelenumii}{\arabic{enumii}.}
\renewcommand{\labelenumiii}{(\alph{enumiii}) }
\newcommand{\ehb}[2]{\overset{\text{#1}}{\underset{\text{#2}}{=}}}
\newcommand{\ec}[1]{\begin{center}\fbox{\begin{minipage}{16.6cm}#1\end{minipage}}\end{center}}
\newcommand{\ecr}[1]{\begin{center}\fbox{\begin{minipage}{15cm}#1\end{minipage}}\end{center}}

\newcommand{\eca}[2]{\vspace{2.3mm}\begin{center}\fbox{\begin{minipage}{#1cm}#2\end{minipage}}\end{center}\vspace{2mm}}
\newcommand{\ecac}[2]{\vspace{2.3mm}\begin{center}\fcolorbox{cyan}{cyan!2}{\parbox{#1cm}{\bf
 #2
 }}\end{center}\vspace{3mm}}
 \newcommand{\ecag}[4]{\vspace{2mm}\begin{center}\fcolorbox{#2}{#2!#3}{\parbox{#1cm}{\bf
 #4
 }}\end{center}\vspace{2mm}}
 %Convergence
 
 
 \newcommand{\biblitest}[1]{\textit{#1}}
 
 \newcommand{\cvu}[1]{\overset{\text{cvu}}{\underset{#1}\longrightarrow}}
\newcommand{\cvs}[1]{\overset{\text{cvs}}{\underset{#1}\longrightarrow}}
\newcommand{\cvw}[1]{\overset{\text{cv weakly}}{\underset{#1}\rightharpoonup}}
\newcommand{\cv}[2]{\overset{\text{cv}}{\underset{#1 \rightarrow #2}\longrightarrow}}

 \newcommand{\ecacnbf}[2]{\vspace{1mm}\begin{center}\fcolorbox{cyan}{cyan!20}{\parbox{#1cm}{
 #2
 }}\end{center}\vspace{1mm}}
\newcommand{\sd}[2]{\sum\limits_{\renewcommand{\arraystretch}{0.4}\left.\begin{array}{c}\scriptstyle{#1} \\\scriptstyle{#2}\end{array}\right.}}
\newcommand{\stt}[3]{\sum\limits_{\renewcommand{\arraystretch}{0.4}\left.\begin{array}{c}\scriptstyle{#1} \\\scriptstyle{#2}\\\scriptstyle{#3} \end{array}\right.}}
\newcommand{\dsd}[2]{\dsum\limits_{\renewcommand{\arraystretch}{0.4}\left.\begin{array}{c}\scriptstyle{#1} \\\scriptstyle{#2}\end{array}\right.}}
\newcommand{\ud}[2]{\bigcup\limits_{\renewcommand{\arraystretch}{0.4}\left.\begin{array}{c}\scriptstyle{#1} \\\scriptstyle{#2}\end{array}\right.}}
\newcommand{\ie}[1]{\left\llbracket#1\right\rrbracket}
\newcommand{\id}[2]{\bigcap\limits_{\renewcommand{\arraystretch}{0.4}\left.\begin{array}{c}\scriptstyle{#1} \\\scriptstyle{#2}\end{array}\right.}}
\newcommand{\di}[2]{_{\renewcommand{\arraystretch}{0.4}\left.\begin{array}{l}\scriptstyle{#1} \\\scriptstyle{#2}\end{array}\right.}}
\newcommand{\equ}[2]{\underset{#1\to#2}{\sim}}
\theoremstyle{definition}

\newtheoremstyle{bobox}{}{}{}{}{}{}{ }%espace obligatoire entre accolades precedentes
{\raisebox{-4pt}{\shadowbox{\magenta{{\thmname{\bf #1}\thmnumber{~\bf#2}}}}}}
\theoremstyle{bobox}

\newcommand{\thec}[1]{\ec{\begin{theo}#1\end {theo}}}
\def\E#1{\left\lfloor#1\right\rfloor}
\newcommand{\bsuma}[3]{\ensuremath\mathop{\scalebox{#1}{$\displaystyle\sum_{#2}^{#3}$}}}
\newcommand{\bproda}[3]{\ensuremath\mathop{\scalebox{#1}{$\displaystyle\prod_{#2}^{#3}$}}}
\newcommand{\binta}[3]{\ensuremath\mathop{\scalebox{#1}{$\displaystyle\int_{#2}^{#3}$}}}
\newcommand{\boplus}[1]{\ensuremath\mathop{\scalebox{#1}{$\displaystyle\bigoplus$}}}
%\newcommand{\ex}[1]{\begin{center}\fbox{\begin{minipage}{16.7cm}\begin{theo}#1\end {theo}\end{minipage}}\end{center}}
\newcommand{\thecr}[1]{\begin{center}\fbox{\begin{minipage}{15cm}\begin{theo}#1\end {theo}\end{minipage}}\end{center}}
\def\rd#1{\uppercase{\text{\cmmd #1}}}
\def\prd#1{\uppercase{\text{\pcmmd #1}}}

\newcommand{\MG}[1]{\marginpar{#1}}
%\newcommand{\NOTE}[1]{(\footnote{#1})}
\newcommand{\GM}{\mathfrak{M}}
\newcommand{\accol}[2]{\left\{\begin{array}{#1}#2\end{array}\right.}
\def\citation#1{\nobreak\bigbreak  {\hbox to \hsize{\hfill\let\\=\cr\null%
 \hfill\vbox{\halign{\it##\hfill\cr#1\hfill\crcr}}}}}
%Mode math : Mettre dans #1 les indications de centrage l/c/r et dans #2 le contenu de
%l'accolade. La syntaxe est par exemple $\accol{ccr}{a&b&c\\d&e&f}$.

 %Guillemets, Gras, italique...
 \newcommand{\eg}[1]{\og #1\fg\,}
 \newcommand{\gi}[1]{\textbf{\textit{#1}}}
 \newcommand{\gieg}[1]{\textbf{\textit{\eg{#1}}}}
\newcommand{\eggi}[1]{\eg{\textbf{\textit{#1}}}}

\newcommand{\pbar}{\rule[-1pt]{0.4pt}{9pt}}   % pbar signifie p(etite) barre
\newcommand{\dpbar}{\pbar\,\pbar}       % dpbar signifie d(ouble) p(etite) barre
%\def\cadre#1{\vbox{\hrule\hbox{\vrule\vbox \spread 2pt{\vfil\hbox \spread 2pt{\hfil#1\hfil}\vfil}\vrule}\hrule}}
%\newcommand{\cadre}{{\vbox{\hrule\hbox{\vrule\vbox \spread 2pt{\vfil\hbox \spread 2pt{\hfil#1\hfil}\vfil}\vrule}\hrule}}}






\newcommand{\scal}[2]{\left<#1\,|\,#2\right>}

%Intervalles
\newcommand{\ioo}[2]{\left(\,#1\,;\,#2\,\right)} % Intervalle ouvert � gauche et � droite, ad lib.
\newcommand{\iof}[2]{\left(\,#1\,;\,#2\,\right]}
\newcommand{\ifo}[2]{\left[\,#1\,;\,#2\,\right)}
\newcommand{\ifmfm}[2]{\left[\,#1\,;\,#2\,\right]}

\newcommand{\fleche}[1]{\overrightarrow{\vphantom{M}#1}}
%\renewcommand{\longmapsto}{\,\epsfbox{d:/tex/fleches.1}\,}
%\renewcommand{\longrightarrow}{\,\epsfbox{d:/tex/fleches.2}\,}
%\newcommand{\etc}[1]{#1\,\dots\,#1}
\newenvironment{marges}[2]{\begin{list}{}{%
\setlength{\topsep}{0pt}%
\setlength{\leftmargin}{0pt}%
\setlength{\rightmargin}{0pt}%
\setlength{\listparindent}{\parindent}%
\setlength{\itemindent}{\parindent}%
\setlength{\parsep}{0pt plus 1pt}%
\addtolength{\leftmargin}{#1}%
\addtolength{\rightmargin}{#2}%
}\item }{\end{list}}
\renewcommand{\leq}{\leqslant}\renewcommand{\geq}{\geqslant}

\def\indic#1{[\,{\em Indication~: #1}\,]}
\newcommand{\dsp}{\displaystyle}

%Trois points SW-NE
\def\adots{\mathinner{\mkern2mu\raise1pt\hbox{.}
\mkern3mu\raise4pt\hbox{.}\mkern1mu\raise7pt\hbox{.}}}

\newcommand{\col}[2]{\begin{pmatrix}#1\\#2\end{pmatrix}}
\newcommand{\ligne}[2]{\begin{pmatrix}#1&#2\end{pmatrix}}
\newcommand{\bloc}[4]{\begin{pmatrix}#1&#2\\#3&#4\end{pmatrix}}


\def\og{\protect\@flqq}
 
\def\@flqq{\relax \ifmmode \ll \else
   {\raise .2ex\hbox{$\scriptscriptstyle \ll $}}\fi}
\def\fg{\protect\@frqq}
\def\@frqq{\relax \ifmmode \gg \else
   {\raise .2ex\hbox{$\scriptscriptstyle \gg $}}\fi}
\makeatother
%
\pagestyle{plain}


%Caract�ristique 
\def\car#1{\chi\dbind{#1}}



\def\vec#1{\overrightarrow{\vphantom A#1}}

\def\rep#1{\ensuremath{\rd r\,_{#1}}}

\def\cit#1{\begin{center}\begin{minipage}{15cm}#1\end{minipage}\end{center}}
%%%%%%% Num�rotation des parties, questions, sous-questions et al.
\newif\ifaffiche\affichetrue
\newcounter{cpart}\newcounter{num}[cpart]\newcounter{sousnum}[num]\newcounter{ssnum}[sousnum]
\newcommand{\partie}[1] {\noindent\refstepcounter{cpart} {\framebox{\large{\quad\textbf{Partie\ \Roman{cpart} #1\quad}}}}\\\\ }
\newcommand{\quest}{\\\noindent\refstepcounter{num}{\textbf{\Roman{cpart}.\Alph{num}.\ }}}
\newcommand{\squest}{\\\refstepcounter{sousnum}\ifaffiche{\textbf{\Roman{cpart}.\Alph{num}\thesousnum.\,}}\else\affichetrue\fi}
\newcommand{\ssquest}{\refstepcounter{ssnum}{\textbf{\Roman{cpart}.\Alph{num}\thesousnum\roman{ssnum}.\,}}}
\newcommand{\questet}{\noindent\refstepcounter{num}{\textbf{\thenum$^*$)\ }}}
% Pour faire r�f�rence � une question dans le texte, du genre : utiliser le I.A.4ii

\def\refpartie#1{\setbox#1=\hbox{\textbf{\Roman{cpart}}}}
\def\refquest#1{\setbox#1=\hbox{\textbf{\Roman{cpart}.\Alph{num}}}}
\def\refsquest#1{\setbox#1=\hbox{\textbf{\Roman{cpart}.\Alph{num}\thesousnum}}}
\def\refssquest#1{\setbox#1=\hbox{\textbf{\Roman{cpart}.\Alph{num}\thesousnum\roman{ssnum}}}}
\def\grefpartie#1#2{\setbox#1#2=\hbox{\textbf{\Roman{cpart}}}}
\def\grefquest#1#2{\setbox#1#2=\hbox{\textbf{\Roman{cpart}.\Alph{num}}}}
\def\grefsquest#1#2{\setbox#1#2=\hbox{\textbf{\Roman{cpart}.\Alph{num}\thesousnum}}}
\def\grefssquest#1#2{\setbox#1#2=\hbox{\textbf{\Roman{cpart}.\Alph{num}\thesousnum\roman{ssnum}}}}
%Norme
\newcommand{\module}[1]{\,\pbar\,#1\,\pbar\,}
\newcommand{\norm}[1]{\,\dpbar\,#1\,\dpbar\,}
\newcommand{\normeinf}[1]{\,\dpbar\,#1\,\dpbar\,_{\infty}}
\newcommand{\subord}[1]{\pbar\,\pbar\,\pbar\ #1\,\pbar\,\pbar\,\pbar\,}

\def\LE{\mathscr{L}(E)}
\def\MR{\mathfrak{M}_n(\R)}
\def\MC{\mathfrak{M}_n(\C)}
\def\diint{\dint\!\!\!\dint\limits_}
\newenvironment{narrow}[2]{%
 \begin{list}{}{%
  \setlength{\topsep}{0pt}%
  \setlength{\leftmargin}{#1}%
  \setlength{\rightmargin}{#2}%
  \setlength{\listparindent}{\parindent}%
  \setlength{\itemindent}{\parindent}%
  \setlength{\parsep}{\parskip}%
 }%
\item[]}{\end{list}}

\newenvironment{lemme}{
\vspace{.5em}
\begin{narrow}{.7cm}{.7cm}
$\left[\begin{tabular}{p{\linewidth}}
}
{
\end{tabular}\right.$\\
\end{narrow}
\vspace{.5em}
}


\usepackage{fancybox}



%   TEXTE : 


%\renewcommand{\labelenumi}{\textbf{\Roman{enumi}-}}

%\renewcommand{\Frlabelitemi}{$\bullet$} for french language
%\renewcommand{\Frlabelitemii}{$\star$}

%\renewcommand{\Frlabelitemiii}{$\-$}
\renewcommand{\labelenumi}{\shadowbox{\red\arabic{enumi}}}
\renewcommand{\labelenumii}{\textbf{\arabic{enumii})} }
\renewcommand{\labelenumiii}{(\alph{enumiii})}

\addtocounter{secnumdepth}{1}
\setcounter{tocdepth}{4}
\renewcommand{\thesubsubsection}{\arabic{chapter}.\arabic{section}.\arabic{subsection}.\alph{subsubsection}}
%   TEXTE :

\def\teh#1#2#3{\overset{\text{ \tiny{#2}}}{\rule{0pt}{#1pt}#3}}
%\renewcommand{\labelenumi}{\textbf{\Roman{enumi}-}}


%\renewcommand{\Frlabelitemi}{$\bullet$} for french language
%\renewcommand{\Frlabelitemii}{$\star$}


%\renewcommand{\Frlabelitemiii}{$\-$}
%\renewcommand{\labelenumi}{\shadowbox{\blue\arabic{enumi}.}}
\renewcommand{\labelenumi}{\textbf{\arabic{enumi})}}
\renewcommand{\labelenumii}{\textbf{\alph{enumii})} }
\renewcommand{\labelenumiii}{(\roman{enumiii})}
\renewcommand{\thechapter}{\arabic{chapter}}
\renewcommand{\thesection}{\arabic{chapter}.\arabic{section}}
\renewcommand{\thesubsection}{\arabic{chapter}.\arabic{section}.\arabic{subsection}}
%\renewcommand{\thesubsubsection}{\arabic{chapter}.\arabic{section}.\arabic{subsection}.\:\raisebox{-6pt}{\shadowbox{\blue\arabic{subsubsection}}\:\:}}

\renewcommand{\thesubsubsection}{\arabic{chapter}.\arabic{section}.\arabic{subsection}.\arabic{subsubsection}}
\addtocounter{secnumdepth}{1}
%\setcounter{tocdepth}{4}
%\renewcommand{\thesubsubsection}{\arabic{chapter}.\arabic{section}.\arabic{subsection}.\alph{subsubsection}}
%   TEXTE :
%\makeindex
\usepackage{fancyhdr}
%\setlength{\headheight}{31pt}%\usepackage{marvosym}
\usepackage{wasysym}%\usepackage{fontspec}
%\newfontfamily\DejaSans{DejaVu Sans}
\pagestyle{fancy}


%
\usepackage{charter}
\fancyfoot{{\small{Cheikh Toure}} \hskip1.3cm\hrulefill\hspace{0.7cm}$\bullet\:\square$ \hspace{0.6cm}page \thepage}



%  \rhead{\begin{Large}\shadowbox{\blue{\textbf{\old{2014}-\old{2015}}}}\end{Large}}
%\renewcommand{\headrulewidth}{0pt}
%\renewcommand{\labelenumi}{\textbf{\Roman{enumi}-}}

%\renewcommand{\Frlabelitemi}{$\bullet$} for french language

\renewcommand{\labelenumi}{\bf\arabic{enumi})}
\renewcommand{\labelenumii}{(\alph{enumii}) }


\usepackage{charter}
\pagestyle{fancy}
\date{}\usepackage{bclogo}
\xdefinecolor{bb}{named}{OrangeRed}
\xdefinecolor{cc}{named}{HotPink}
\xdefinecolor{dd}{named}{MediumSpringGreen}
\xdefinecolor{ee}{named}{LightSeaGreen}
\xdefinecolor{ff}{named}{MidnightBlue}
\fancyfoot{{\small{Cheikh Tour�}} \hskip 0.7cm\hrulefill\hspace{0.8cm} page \thepage\:
$\bullet\,\square$\:}

\def\pr{\begin{center}\bf \blue Proof :\end{center}}
%   \newtheorem{Lemme}{Lemma}
%\newtheorem{prop}{Proposition}
%\newtheorem{definition}{Definition}
%\newtheorem{theo}{Theorem}
%\newtheorem{coro}{Corollary}

\newtheorem{theorem}{Theorem}[section]
\newtheorem{lemma}[theorem]{Lemma}
\newtheorem{proposition}[theorem]{Proposition}
\newtheorem{corollary}[theorem]{Corollary}
\newtheorem{definition}[theorem]{Definition}
\newtheorem{notation}[theorem]{Notation}
%\newtheorem{remark}[theorem]{Remark}
\def\remark{\gi{Remark} : \\}


\def \argmin{\arg \hspace{-0.05cm} \min}
\def \argmax{\arg \hspace{-0.05cm} \max}
\newcommand{\HRule}{\rule{\linewidth}{0.5mm}}


\titleformat{\chapter}{\normalfont\huge}{\thechapter.}{20pt}{\huge\it}

 \newcommand{\pourRef}[1]{\textbf{\textit{ #1}}}

\newcommand{\ba}{\begin{eqnarray}}
\newcommand{\ea}{\end{eqnarray}}
\newcommand{\baStar}{\begin{eqnarray*}}
\newcommand{\eaStar}{\end{eqnarray*}}



\usepackage[colorinlistoftodos,bordercolor=orange,backgroundcolor=orange!20,linecolor=orange,textsize=scriptsize]{todonotes}
\newcommand{\rob}[1]{\todo[inline]{\textbf{Robert: }#1}}

\title{New stochastic sketching methods for Big Data Ridge Regression}
\author{Cheikh Saliou Tour\'e \\ \\
Student at ENS Cachan\\\\
Tutor : Robert Gower \\ \\
Inria Paris (Sierra department)\\\\ }




\date{July, 2017}


\begin{document}

\renewcommand\bibname{References}
%\renewcommand\contentsname{Table of contents}
\maketitle


\begin{abstract}

//

\end{abstract}
\tableofcontents
\newpage

\chapter{General Sketching method}

$A$ is a $n \times n$ positive definite matrix representing our problem.\\ 
$s$ is the sketch size.\\
 $\acco{S_{i}}_{i=1,\dots,r}$ is the set of $r$ realizations of our $s\times n$ sketch matrix.\\
We denote by $S$ the $s\times n$ random sketch matrix, which is such that $S = S_{i}$ with probability $p_{i}$. \\ 
 
Throughout the computations, we denote by $Z = A S^{T} (S A S^{T})^{-1} S A$. That is a quantity that intervenes in the computation of the convergence rate\footnote{will put before the intervention of the convergence rate in the convergence of our sequence to the optimal solution }.\\



The convergence rate is defined by $\rho = 1 - \lambda_{min}(A^{-\frac12}E[Z]A^{-\frac12}  )$.\\

By defiition, $A^{-\frac12}E[Z]A^{-\frac12} = \dsp\sum\limits_{i} p_{i} A^{\frac 12} S_{i}^{T} (S_{i}  A  S_{i}^{T})^{-1} S_{i} A^{\frac 12}$ \\
for any $i \in \acco{1,\dots,n}$, $A^{\frac 12} S_{i}^{T} (S_{i}  A  S_{i}^{T})^{-1} S_{i} A^{\frac 12}$ is a projection matrix (a matrix such that $M^{2} = M$) and then its eigenvalues are a nonempty subset of $\acco{0,1}$.\\

Since $\lambda_{max}$ is a convex function, we obtain that :\\

$0 \leq \lambda_{min}(A^{-\frac12}E[Z]A^{-\frac12}) \leq  \lambda_{max}(A^{-\frac12}E[Z]A^{-\frac12}) \leq \dsp\sum\limits_{i} p_{i} \lambda_{max}(A^{\frac 12} S_{i}^{T} (S_{i}  A  S_{i}^{T})^{-1} S_{i} A^{\frac 12}) \leq 1$.\\

Denote by $\bold{C} = (S_{1}^{T},\dots,S_{r}^{T})$ which is of size $ n \times r s$.\\
  \ecag{11}{blue}{19}{
\begin{lemma} \label{general lemma}

$A^{-\frac12}E[Z]A^{-\frac12} = (A^{\frac 12} \bold{C} D)(D \bold{C}^{T} A^{\frac 12})$ where \\$D =  \,\text{diag}(\sqrt{p_{1}} (S_{1}A S_{1}^{T})^{-\frac 12},\dots, \sqrt{p_{r}}(S_{r}A S_{r}^{T})^{-\frac 12}) \in \M_{r s}(\R).$ Plus :
$\\\\$
$\dsp \lambda_{min}(A^{-\frac12}E[Z]A^{-\frac12} )  \geq  \dsp  \frac{\lambda_{min}(A) \lambda_{min}(\bold{C} \bold{C}^{T} )}{ \lambda_{max}(A)} \min_{i} \frac{p_{i}}{\lambda_{max}(S_{i}^{T} S_{i})} $

\end{lemma}
}
\pr
$A^{-\frac12}E[Z]A^{-\frac12} = \dsp\sum\limits_{i} p_{i} A^{\frac 12} S_{i}^{T} (S_{i}  A  S_{i}^{T})^{-1} S_{i} A^{\frac 12}$ \\
Then we straightforwardly obtain that : $A^{-\frac12}E[Z]A^{-\frac12} = A^{\frac 12} \bold{C} D^{2} \bold{C}^{T} A^{\frac 12}$.\\

$\lambda_{min}(A^{-\frac12}E[Z]A^{-\frac12} ) \geq \lambda_{min}(\bold{C}^{T}A \bold{C} ) \lambda_{min}(D^{2})$

$\dsp \lambda_{min}(D^{2}) =  \min_{i}  \frac{p_{i}}{\lambda_{max}(S_{i} A S_{i}^{T} ) } \geq  \min_{i}\frac{p_{i} }{\lambda_{max}(S_{i}^{T} S_{i}) \lambda_{max}(A)}.$\\
 Therefore, 
$\dsp \lambda_{min}(A^{-\frac12}E[Z]A^{-\frac12} ) \geq  \min_{i}  \frac{p_{i} \lambda_{min}(\bold{C}^{T}A \bold{C} )}{\lambda_{max}(S_{i}^{T} S_{i}) \lambda_{max}(A) }  =  \frac{\lambda_{min}(A) \lambda_{min}(\bold{C} \bold{C}^{T} )}{ \lambda_{max}(A)} \min_{i} \frac{p_{i}}{\lambda_{max}(S_{i}^{T} S_{i})}$
$\bullet$



\chapter{Block Coordinate Descent Method} \label{newton}




\section{Algorithm}

\section{Convergence rate}

$A$ is a $n \times n$ positive definite matrix representing our problem.\\ 
For any subset $C$ of $\acco{1,\dots,n}$ of length $s$, we denote by $I_{C}$ the $s\times n$ matrix which rows are $\acco{e_{i}^{T}}_{i\in C}$ up to a permutation, where $\acco{e_{i}}_{i=1,\dots,n}$ is a canonical basis of $\R^{n}.$\\ 
Denote by $\acco{C_{i}}_{i=1,\dots,r}$ the subsets of $\acco{1,\dots,n}$ of size $s$ : that implies that $\dsp r \egaldef  \pare{\substack{n \\ s}}.$
 
Throughout the computations, we denote by $Z = A I_{C}^{T} (I_{C} A I_{C}^{T})^{-1} I_{C} A$.\\

The convergence rate is defined by $\rho = 1 - \lambda_{min}(A^{-\frac12}E[Z]A^{-\frac12}  )$.\\


Denote by $\bold{C} = (I_{C_{1}}^{T},\dots,I_{C_{r}}^{T})$ which is of size $ n \times r s$.\\
 
 By \textbf{lemma} \ref{general lemma}, we have that : 
$\dsp \lambda_{min}(A^{-\frac12}E[Z]A^{-\frac12} )  \geq  \dsp  \frac{\lambda_{min}(A) \lambda_{min}(\bold{C} \bold{C}^{T} )}{ \lambda_{max}(A)} \min_{i} \frac{p_{i}}{\lambda_{max}(I_{C_{i}}^{T} I_{C_{i}})} $
 
For any $i\in \acco{1,\dots,n}$, for any $x$ in $\R^{n}$,
$\scal{I_{C_{i}}^{T} I_{C_{i}} x }{x} =$
$ \norm{ I_{C_{i}}x }^{2} \leq \norm{x}^{2}$, then $\lambda_{max}( I_{C_{i}}^{T} I_{C_{i}}  ) \leq 1$.\\
 
 Therefore, 
$\dsp \lambda_{min}(A^{-\frac12}E[Z]A^{-\frac12} ) \geq \frac{ \lambda_{min}(A) \lambda_{min}(\bold{C} \bold{C}^{T} )}{\lambda_{max}(A)}  \min_{i} p_{i}.$\\

$\bold{C} \bold{C}^{T}= \dsp\sum\limits_{i=1}^{r} I_{C_{i}}^{T} I_{C_{i}} = \pare{\substack{n-1 \\ s-1}} I_{n} $ and then we obtain that corollary :
 
 \ecag{17}{green}{19}{
\begin{corollary} \label{coordinate}
$\\\\$
$\dsp \lambda_{min}(A^{-\frac12}E[Z]A^{-\frac12} )  \geq  \dsp \pare{\substack{n-1 \\ s-1}}\frac{\lambda_{min}(A)}{\lambda_{max}(A)}\min_{i} p_{i} $.\\

If we choose $\acco{p_{i}}_{i=1}^{r}$ as the uniform probability of choosing $s$ rows uniformly on $\acco{1,\dots,n}$, $i.e.$ for any $i$, $p_{i} = \dsp\frac{1}{\pare{\substack{ n \\ s }}}$, then :\\\\
$\dsp \lambda_{min}(A^{-\frac12}E[Z]A^{-\frac12} ) \geq \frac{s}{n} \frac{\lambda_{min}(A)}{\lambda_{max}(A)} $
\end{corollary}
}


 \rob{This is already pretty interesting! It shows an improvement for using bigger bachsize! We should try to push this further, for instance, when $s =n$ we know the method converges in one step. It would be great if we have a convergence rate that shows this phenomena. In other words, when $s =n$ we have $\lambda_{\min}(A^{-1/2}E[Z]A^{-1/2}) =1$ ! Also, please have a look at the paper ``paving\_kaczmarz.pdf'' which I've just added to our repo.}
 
 %\subsection{A convenient probability}
 
 %Suppose here that $\dsp p_{i} = \frac{Tr( I_{C_{i}} A I_{C_{i}}^{T} ) }{\norm{A^{\frac12}\bold{C}}^{2}_{F} }$, for any $i = 1,\dots,r.$\\

 

\chapter{Randomized orthonormal systems}

This type of randomized sketch is well-suited for big data regression, thanks to the efficiency of matrix multiplication used in this method.\\
When the dimension of our matrix $A$ is $n$, we denote by $H_{n}$ the Hadamard matrix (well defined if the dimension of the problem $n$ is a power of $2$) defined recursively as :\\

$H_{2^{p}} = \begin{pmatrix} H_{2^{p-1}} & - H_{2^{p-1}} \\
					H_{2^{p-1}} & H_{2^{p-1}}  \end{pmatrix} $ for $p=1,2,\dots$and $H_{1} = 1.$\\

The Hadamard sketch consists of choosing a random sketch matrix $S \in \M_{s,n}$ where $s$ is the sketch size of the problem, as follows :\\ 
we sample $s$ $i.i.d.$ rows of the form $s^{T} = e_{j}^{T}H_{n} D $ with probability $\frac 1n$ for $j = 1,\dots,n$,where $(e_{j})_{j}$ forms a canonical basis of $\R^{n}$, and $D = diag(\nu)$ is a diagonal matrix of $i.i.d.$ Rademacher variables $\nu \in \acco{-1,1}^{n}$.  


\section{Algorithm}


\section{Convergence rate}



Now we denote by $Z = A S^{T} (S A S^{T})^{-1} S A$, where $S$ is our Hadamard random matrix.\\
For any subset $C$ of $\acco{1,\dots,n}$ of length $s$, we denote by $I_{C}$ the $s\times n$ matrix which rows are $\acco{e_{i}^{T}}_{i\in C}$ up to a permutation, where $\acco{e_{i}}_{i=1,\dots,n}$ is a canonical basis of $\R^{n}.$\\ 

By construction, $S = I_{C} H D$ where $C$ is a uniform random subset of $\acco{1,\dots,n}$ of size $s$,  $H$ is the $Hadamard$ matrix ($H H^{T} = n I_{n}$) and $D = diag(\nu)$ is a diagonal matrix of $i.i.d.$ Rademacher variables $\nu \in \acco{-1,1}^{n}$. \\

Recall that the convergence rate is  $\rho = 1 - \lambda_{min}(A^{-\frac12}E[Z]A^{-\frac12}  )$. From \textbf{lemma} \ref{general lemma}, we have that :

\ecag{11}{green}{19}{
\begin{corollary} [Unifom sketching]
$\\\\$
$\dsp \lambda_{min}(A^{-\frac12}E[Z]A^{-\frac12} ) \geq \frac{s}{n} \frac{\lambda_{min}(A)}{\lambda_{max}(A)} $

\end{corollary}
}
\pr

Let's condition on the \emph{Rademacher diagonal matrix} $D$.\\

Define by $\tilde{A}_{D} = \frac{H} {\sqrt{n}} D A D \frac{H^{T}}{\sqrt{n}}$. We obtain that :

\baStar
A^{-\frac12}E[Z|D]A^{-\frac12} &=& E[A^{\frac 12} S^{T} (S A S^{T})^{-1} S A^{\frac 12}|D ] \\
&=& \dsp\sum\limits_{i} p_{i} A^{\frac 12} D H^{T} I_{C_{i}}^{T} (I_{C_{i}} H D A D H^{T} I_{C_{i}}^{T})^{-1} I_{C_{i}} H D A^{\frac 12} \\
&=& \frac1n A^{\frac 12}D H^{T} E[ I_{C}^{T} (I_{C} \tilde{A}_{D} I_{C}^{T})^{-1} I_{C} ] HD A^{\frac 12} \\
&=& D H^{-1} HD \frac1n A^{\frac 12}D H^{T} E[ I_{C}^{T} (I_{C} \tilde{A}_{D} I_{C}^{T})^{-1} I_{C} ] HD A^{\frac 12} D H^{T} (H^{T})^{-1}D  \\
&=& D H^{-1} \tilde{A}^{\frac12}_{D} E[ I_{C}^{T} (I_{C} \tilde{A}_{D} I_{C}^{T})^{-1} I_{C} ]\tilde{A}^{\frac12}_{D}\,  n (H^{T})^{-1}D \\
&=& D H^{-1} \tilde{A}^{\frac12}_{D} E[ I_{C}^{T} (I_{C} \tilde{A}_{D} I_{C}^{T})^{-1} I_{C} ]\tilde{A}^{\frac12}_{D} \,H D \\
  \eaStar
  
Hence :\\

$\lambda_{min}(A^{-\frac12}E[Z]A^{-\frac12}) = \lambda_{min}\pare{E_{D}\croc{D H^{-1} \tilde{A}_{D}^{\frac 12} E[ I_{C}^{T} (I_{C} \tilde{A}_{D} I_{C}^{T})^{-1} I_{C}] \tilde{A}_{D}^{\frac 12} H D}}$.\\
Denote by $(D_{i})_{i=1,\dots,2^{n}}$ the $2^{n}$ possible values of the random matrix $D$.\\
We obtain that :\\

$\lambda_{min}(A^{-\frac12}E[Z]A^{-\frac12}) = \lambda_{min}\pare{ \dsp\sum\limits_{i=1}^{2^{n}} \frac{1}{ 2^{n}} D_{i} H^{-1} \tilde{A}_{D_{i}}^{\frac 12} E[ I_{C}^{T} (I_{C} \tilde{A}_{D_{i}} I_{C}^{T})^{-1} I_{C} ] \tilde{A}_{D_{i}}^{\frac 12} H D_{i}}$.\\
And thanks to the concavity of $\lambda_{min}$, we obtain that :
\baStar
\lambda_{min}(A^{-\frac12}E[Z]A^{-\frac12}) &\geq&  \dsp \dsp\sum\limits_{i=1}^{2^{n}} \frac{1}{ 2^{n}} \lambda_{min}\pare{D_{i} H^{-1} \tilde{A}_{D_{i}}^{\frac 12} E[ I_{C}^{T} (I_{C} \tilde{A}_{D_{i}} I_{C}^{T})^{-1} I_{C} ] \tilde{A}_{D_{i}}^{\frac 12} H D_{i}}\\
 &=& \sum\limits_{i=1}^{2^{n}} \frac{1}{ 2^{n}} \lambda_{min}\pare{ \tilde{A}_{D_{i}}^{\frac 12} E[ I_{C}^{T} (I_{C} \tilde{A}_{D_{i}} I_{C}^{T})^{-1} I_{C} ] \tilde{A}_{D_{i}}^{\frac 12}}
\eaStar

We then straightforwardly use the uniform case in \textbf{Corollary} \ref{coordinate} to obtain that :\\

$\lambda_{min}(A^{-\frac12}E[Z]A^{-\frac12}) \geq  \dsp\sum\limits_{i=1}^{2^{n}} \frac{1}{ 2^{n}} \frac{s}{n} \frac{\lambda_{min}(\tilde{A}_{D_{i}})}{\lambda_{max}(\tilde{A}_{D_{i}})}$.\\
For all $i = 1,\dots, 2^{n}$, $\tilde{A}_{D_{i}}$ is similar to $A$, and then finally :\\

$\dsp \lambda_{min}(A^{-\frac12}E[Z]A^{-\frac12}) \geq \frac{s}{n} \frac{\lambda_{min}(A)}{\lambda_{max}(A)}$ $\bullet$




\chapter{Count-min Sketches}

%\ecag{11}{green}{19}{
%\begin{definition}
%
%
%
%\end{definition} 
%}

\section{Algorithm}


\section{Convergence rate}

%$S$ is constructed as follows :\\
%For every $i\in\acco{1,\dots,n}$, $l$ is chosen uniformly on $\acco{1,\dots,n}$ and $\epsilon$ uniformly on $\acco{-1,1}$, then $S$ is updated in his $l^{th}$ row as :\\
%$S(l, :) := S(l,:) + \epsilon \, e_{i}^{T}$, where $e_{i}$ is the $i^{th}$ coloumn of the identity matrix.\\\\

Denote by $\pare{e_{i}}_{i=1,\dots,n}$ a canonical basis of $\R^{n}$  and $\pare{f_{i}}_{i=1,\dots,s}$ a canonical basis of $\R^{s}.$\\ 
Then we obtain that every count-min random matrix is of the form : \\
$\dsp S = \dsp\sum_{i=1}^{n} \dsp\epsilon(i) f_{\pi(i)}e_{i}^{T} \in \M_{s,n}(\R)$, where $\epsilon : \foncfleche{\acco{1,\dots,n}}{\acco{1,-1}}$ and $\pi : \foncfleche{\acco{1,\dots,n}}{\acco{1,\dots,s}} $.\\

We therefore can rewrite $S$ as :\\
 $S = \pare{ \epsilon(1)f_{\pi(1)},\epsilon(2)f_{\pi(2)},\dots,\epsilon(n)f_{\pi(n)} } \begin{pmatrix} e_{1}^{T} \\ \vdots \\ e_{n}^{T} \end{pmatrix} = \pare{ f_{\pi(1)}, f_{\pi(2)},\dots, f_{\pi(n)} } \text{diag}\pare{\epsilon(1),\dots,\epsilon(n)}$.\\
 

For any $\pi : \foncfleche{\acco{1,\dots,n}}{\acco{1,\dots,s}} $, define by $f_{\pi}$ the $s\times n$ matrix $\pare{ f_{\pi(1)}, f_{\pi(2)},\dots, f_{\pi(n)} }.$\\

Let $S$ be a random count-min sketch matrix.\\
 $S = f_{\pi} D$ where $\pi$ is a uniform random element of $\acco{1,\dots,s}^{\acco{1,\dots,n}}$ and $D = diag(\nu)$ is a diagonal matrix of $i.i.d.$ Rademacher variables $\nu \in \acco{-1,1}^{n}$. \\

Denote again by $Z = A S^{T} (S A S^{T})^{-1} S A$, where $S$ is our count-min random matrix.\\
Recall that the convergence rate is  $\rho = 1 - \lambda_{min}(A^{-\frac12}E[Z]A^{-\frac12}  )$.\\\\

Denote $r \egaldef s^{n}$ and $\acco{\pi_{1},\dots,\pi_{r}}$ the elements of $\acco{1,\dots,s}^{\acco{1,\dots,n}}$ which is of size $r = s^{n}$.\\
Then, $\pi = \pi_{k}$ with probability $p_{k} \egaldef s^{-n}$.\\
Denote by $\bold{C} = (f_{\pi_{1}}^{T},\dots,f_{\pi_{r}}^{T})$ which is a $ n \times r s$ matrix.\\


 
 \ecag{11}{green}{19}{
\begin{corollary} \label{general}
$\\\\$
$\dsp \lambda_{min}(A^{-\frac12}E[Z]A^{-\frac12} )  \geq  \dsp
\frac{(s-1)\,\lambda_{min}(A)}{n\,s\,\lambda_{max}(A)}$

\end{corollary}
}
\pr 

Denote by $\tilde{A} = D A D.$\\

\baStar
A^{-\frac12}E[Z|D]A^{-\frac12} &=& E[A^{\frac 12} S^{T} (S A S^{T})^{-1} S A^{\frac 12}|D ] \\
&=& \dsp\sum\limits_{i} p_{i} A^{\frac 12} D f_{\pi_{i}}^{T} (f_{\pi_{i}}  D A D f_{\pi_{i}}^{T})^{-1} f_{\pi_{i}} D A^{\frac 12} \\
&=& A^{\frac 12}D E[ f_{\pi}^{T} (f_{\pi} \tilde{A}_{D} f_{\pi}^{T})^{-1} f_{\pi} ] D A^{\frac 12} \\
&=& D \tilde{A}^{\frac 12}_{D} E[ f_{\pi}^{T} (f_{\pi} \tilde{A}_{D} f_{\pi}^{T})^{-1} f_{\pi} ] \tilde{A}^{\frac 12}_{D} D \\
   \eaStar
   
   Then :\\

$\lambda_{min}(A^{-\frac12}E[Z]A^{-\frac12}) = \lambda_{min}\pare{E_{D}\croc{D  \tilde{A}_{D}^{\frac 12} E[ f_{\pi}^{T} (f_{\pi} \tilde{A}_{D} f_{\pi}^{T})^{-1} f_{\pi}] \tilde{A}_{D}^{\frac 12} D}}$.\\
Denote again by $(D_{i})_{i=1,\dots,2^{n}}$ the $2^{n}$ possible values of the random matrix $D$.\\
We obtain that :\\

$\lambda_{min}(A^{-\frac12}E[Z]A^{-\frac12}) = \lambda_{min}\pare{ \dsp\sum\limits_{i=1}^{2^{n}} \frac{1}{ 2^{n}} D_{i}  \tilde{A}_{D_{i}}^{\frac 12} E[ f_{\pi}^{T} (f_{\pi} \tilde{A}_{D_{i}} f_{\pi}^{T})^{-1} f_{\pi} ] \tilde{A}_{D_{i}}^{\frac 12} D_{i}}$.\\
And thanks to the concavity of $\lambda_{min}$, we obtain that :\\

\baStar
\lambda_{min}(A^{-\frac12}E[Z]A^{-\frac12}) &\geq&  \dsp\sum\limits_{i=1}^{2^{n}} \frac{1}{ 2^{n}} \lambda_{min}\pare{D_{i} \tilde{A}_{D_{i}}^{\frac 12} E[ f_{\pi}^{T} (f_{\pi} \tilde{A}_{D_{i}} f_{\pi}^{T})^{-1} f_{\pi} ] \tilde{A}_{D_{i}}^{\frac 12} D_{i}}\\
 &=& \sum\limits_{i=1}^{2^{n}} \frac{1}{ 2^{n}} \lambda_{min}\pare{ \tilde{A}_{D_{i}}^{\frac 12} E[ f_{\pi}^{T} (f_{\pi} \tilde{A}_{D_{i}} f_{\pi}^{T})^{-1} f_{\pi} ] \tilde{A}_{D_{i}}^{\frac 12}}
\eaStar
   
Then by \textbf{lemma} \ref{general lemma} :\\

\baStar
\dsp \lambda_{min}(A^{-\frac12}E[Z]A^{-\frac12} )  &\geq&  \dsp   \sum\limits_{i=1}^{2^{n}} \frac{1}{ 2^{n}} \frac{\lambda_{min}(\tilde{A}_{D_{i}}) \lambda_{min}(\bold{C} \bold{C}^{T} )}{ \lambda_{max}(\tilde{A}_{D_{i}})} \min_{k} \frac{p_{k}}{\lambda_{max}(f_{\pi_{k}}^{T} f_{\pi_{k}})} \\
&=& \frac{\lambda_{min}(A) \lambda_{min}(\bold{C} \bold{C}^{T} )}{ \lambda_{max}(A)} \min_{k} \frac{p_{k}}{\lambda_{max}(f_{\pi_{k}}^{T} f_{\pi_{k}})} \\
\eaStar

Recall that $p_{k} = s^{-n}$ for any $k\in \acco{1,\dots,r}$.\\

For any $x$ in $\R^{n}$, for any $k\in \acco{1,\dots,r}$,

$\scal{f_{\pi_{k}}^{T} f_{\pi_{k}} x }{x} = \norm{ f_{\pi_{k}}x }^{2} = \norm{\dsp\sum_{i=1}^{n} x_{i}f_{\pi_{k}(i)} }^{2} \leq \pare{\dsp\sum_{i=1}^{n} \abs{x_{i}}}^{2} \leq n\norm{x}^{2}$
 and then $\lambda_{max}( f_{\pi_{k}}^{T} f_{\pi_{k}}  ) \leq n$.\\\\
 
 
 
$\bold{C} \bold{C}^{T}= \dsp\sum\limits_{k=1}^{r} f_{\pi_{k}}^{T} f_{\pi_{k}} = s^{n-1}
    \left(
    \begin{array}{ccccc}
    s                                    \\
      & s             &   & \text{\huge1}\\
      &               & \ddots               \\
      & \text{\huge1} &   & s            \\
      &               &   &   & s
    \end{array}
    \right).$\\
Denote by $\dsp M = \frac{1}{s^{n-1}}\, \bold{C} \bold{C}^{T}.$\\

By subtracting $(s-1)I_{n}$ from $M$, we recognize that $s-1$ is an eigenvalue of $M$ with multiplicity $n-1$. Then the trace of $M$ gives us that $n+s-1$ is the other eigenvalue of $M$.\\ 
Hence, $\lambda_{min}(\bold{C} \bold{C}^{T}) = (s-1)s^{n-1}$.\\

Thereby we obtain that :\\ 


$\dsp \lambda_{min}(A^{-\frac12}E[Z]A^{-\frac12} )  \geq  \frac{\lambda_{min}(A) (s-1)s^{n-1}}{ \lambda_{max}(A)}  \frac{s^{-n}}{n}$ \\


$\dsp \lambda_{min}(A^{-\frac12}E[Z]A^{-\frac12} )  \geq  \frac{(s-1)\,\lambda_{min}(A)}{n \, s \,\lambda_{max}(A)} $ $\bullet$


\section{Sparse Shuffling (Spashu)}
\rob{I was calling this Radamacher sketch before, but in truth it is not the Radamacher sketch. So we need to give this a new name. How about Sparse Shuffling Sketch? Or a Spashu sketch for short :) }

Let $\phi: \{1,\ldots, n\} \rightarrow \{1,\ldots, n\}$ be a permutation, selected uniformly at random for all the $n!$ possible permutations. Let $s\in \N$ be an integer that divides $n$, that is, there exists $m \in \N$ such that $n = ms.$ We define $S \in \R^{n \times s}$ as a $s\times n$ Sparse Shuffling sketch when
 \[S = \dsp\sum_{i=1}^s f_i \sum_{j=1+m(i-1)}^{mi} \epsilon(j)e_{\phi(j)}^\top.\]
 Note that there are exactly $m$ non-zero elements in each row of $S$.
 
 We can also define a subsampled Spashu by considering $m \in \N$ as a free parameter such that $m \leq \lfloor \frac{n}{s} \rfloor.$\\\\
 
 
Notice that $S$ can be rewriting as : $S = \dsp\sum_{j=1}^{n} \epsilon_{j}f_{\pi(j)}e_{\phi(j)}^{T} $, where $\pi$ is the function $\fonc{\acco{1,\dots,n}}{\acco{1,\dots,s}}{j}{-\lfloor -\frac{j}{m} \rfloor}.$\\
 
 $\pi$ verifies that for all $i \in \acco{1,\dots,s}$, for all $j \in \acco{1+(m-1) i,\dots,m i}$, $\pi(j) = i. $
 
 For any permutation $\phi$ on $\acco{1,\dots,n}$, denote by $P_{\phi}$ the $n\times n$ matrix $ \begin{pmatrix} e_{\phi(1)}^{T} \\ \vdots \\ e_{\phi(n)}^{T} \end{pmatrix} $.\\
 Denote by $\phi_{1},\dots,\phi_{n!}$ the different permutations of $\mathfrak{S}_{n}$ and define $(p_{k})_{k=1,\dots,n!}$ such that $p_{k} = \frac{1}{n!}$ for all $k$.\\ Let's consider that uniform probability on $\mathfrak{S}_{n}$.\\ Then $\phi = \phi_{k}$ with probability $\dsp \frac{1}{n!}$.\\\\
 
 Let $\epsilon$ be a uniform random vector of $\acco{-1,1}^{n}$ and $\phi$ a uniform random permutation of $\mathfrak{S}_{n}$.\\
Let $S$ be a random shuffling sketch such that : $S = \dsp\sum_{j=1}^{n} \epsilon_{j}f_{\pi(j)}e_{\phi(j)}^{T} .$\\

Denote by $f_{\pi} = \pare{ f_{\pi(1)}, f_{\pi(2)},\dots, f_{\pi(n)} }$ and $D = \text{diag}\pare{\epsilon(1),\dots,\epsilon(n)}$.\\
We have that :\\
$S = \pare{ \epsilon(1)f_{\pi(1)},\epsilon(2)f_{\pi(2)},\dots,\epsilon(n)f_{\pi(n)} } \begin{pmatrix} e_{\phi(1)}^{T} \\ \vdots \\ e_{\phi(n)}^{T} \end{pmatrix} = \pare{ f_{\pi(1)}, f_{\pi(2)},\dots, f_{\pi(n)} } \text{diag}\pare{\epsilon(1),\dots,\epsilon(n)} P_{\phi}.$\\
Then : $S = f_{\pi} D P_{\phi}$.\\

Denote by $\bold{C}_{D} = (\pare{P_{\phi_{1}}^{T} D f_{\pi}^{T},\dots,P_{\phi_{r}}^{T} D f_{\pi}^{T} } $ which is a $ n \times n! \,n$ matrix.\\


Recall that $Z = A S^{T} (S A S^{T})^{-1} S A$, where $S$ is our sparse shuffling random matrix, and that the convergence rate is  $\rho = 1 - \lambda_{min}(A^{-\frac12}E[Z]A^{-\frac12}  )$.\\

 \ecag{11}{green}{19}{
\begin{corollary} \label{shuffle}
$\\\\$
$\dsp \lambda_{min}(A^{-\frac12}E[Z]A^{-\frac12} )  \geq  \dsp
%\frac{(s-1)\,\lambda_{min}(A)}{n\,s\,\lambda_{max}(A)}
$

\end{corollary}
}
\pr

The \textbf{lemma}\ref{general lemma} gives us that :\\ 
$\dsp \lambda_{min}(A^{-\frac12}E\croc{Z|D}A^{-\frac12} ) \geq \dsp  \frac{\lambda_{min}(A) \lambda_{min}(\bold{C}_{D} \bold{C}_{D}^{T})}{ \lambda_{max}(A)} \min_{k} \frac{p_{k}}{\lambda_{max}( P_{\phi_{k}}^{T} D f_{\pi}^{T} f_{\pi} D P_{\phi_{k}} )}.$\\
For all $k = 1,\dots,n!$, $p_{k} = \frac{1}{n!}$ and $P_{\phi_{k}}$ is an orthogonal matrix ( $i.e.$ $P_{\phi_{k}} P_{\phi_{k}}^{T} = I_{n}$). Therefore one obtains that :\\

$\dsp \lambda_{min}(A^{-\frac12}E\croc{Z|D}A^{-\frac12} ) \geq \dsp  \frac{\lambda_{min}(A) \lambda_{min}(\bold{C}_{D} \bold{C}_{D}^{T})}{ n! \, \lambda_{max}(A) \lambda_{max}(f_{\pi}^{T} f_{\pi})}. $\\
 
 \baStar
 \bold{C}_{D} \bold{C}_{D}^{T} &=& \dsp\sum\limits_{k=1}^{n!} P_{\phi_{k}}^{T} D f_{\pi}^{T} f_{\pi} D P_{\phi_{k}} \\
&=& \small  (n-1)!\left(
    \begin{array}{ccccc}
    \tr(f_{\pi}^{T} f_{\pi})                                    \\
      & \tr(f_{\pi}^{T} f_{\pi})               &   & \textbf{\Large $\frac{\tr\pare{D f_{\pi}^{T} f_{\pi} D (J-I_{n}) }}{n-2}$  }\\
      &               & \ddots               \\
      &\textbf{\Large $\frac{\tr\pare{D f_{\pi}^{T} f_{\pi} D (J-I_{n}) }}{n-2}$  }&   & \tr(f_{\pi}^{T} f_{\pi})               \\
      &               &   &   & \tr(f_{\pi}^{T} f_{\pi})   
    \end{array}
    \right)\\
    \eaStar
    Denote by $\lambda_{1} = (n-1)!\tr(f_{\pi}^{T} f_{\pi}) -(n-2)! \tr\pare{D f_{\pi}^{T} f_{\pi} D (J-I_{n}) }     $ and \\$\lambda_{2} = (n-1)!(n-1)\tr(f_{\pi}^{T} f_{\pi})+(n-2)! \tr\pare{D f_{\pi}^{T} f_{\pi} D (J-I_{n}) } $.\\
    By subtracting $\lambda_{1}I_{n}$  from $\bold{C}_{D} \bold{C}_{D}^{T} $, we straightforwardly observe that $\lambda_{1}$ is an eigenvalue of $\bold{C}_{D} \bold{C}_{D}^{T} $ of multiplicity $n-1$. And then taking the trace shows that $\lambda_{2}$ is the remaining eigenvalue.\\
    Hence, $\lambda_{min}( \bold{C}_{D} \bold{C}_{D}^{T} ) = (n-1)!\tr(f_{\pi}^{T} f_{\pi}) -(n-2)! \tr\pare{D f_{\pi}^{T} f_{\pi} D (J-I_{n}) }$
 
\chapter{Conclusion}

\appendix
\begin{thebibliography}{1}

\bibitem{}
{\sc Robert Gower and Peter Richtarik}, {\em Randomized iterative methods for linear systems}, SIAM, 
  (2015).



\end{thebibliography}

\end{document}



